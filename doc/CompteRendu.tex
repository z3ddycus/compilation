\documentclass[hidelinks, a4paper,11pt,twoside,final]{article}

\usepackage[english,francais]{babel}
\usepackage[utf8]{inputenc}
\usepackage{geometry}
\usepackage[T1]{fontenc}
\usepackage[pdftex]{graphicx}
\usepackage{adjustbox}
\usepackage{color}
\usepackage{setspace}
\usepackage{hyperref}
\usepackage[french]{varioref}
\usepackage{comment}

%Opening
\title{\bfseries Compte Rendu \\ Compilation}
\geometry{hmargin=2.5cm,vmargin=3cm}
\begin{document}
\maketitle
\begin{center}
\begin{tabular}{ll}
  Version~: & 0.1\\[.5em]
  Date~: & \date{\today}\\[.5em]
  Rédigé par~: & Thomas \textsc{Capet}\\
               & Yohann \textsc{Henry}\\
\end{tabular}
\end{center}

\newpage

%Table of contents
\newpage
\tableofcontents

%Contents
\newpage

\section{Introduction}
Le projet consiste en la création d'un exécutable permettant la gestion de références contenues dans des fichiers \texttt{bibtex} ou \texttt{LaTeX}.
L'exécutable devait notament extraire les clés cités d'un fichier d'extension \texttt{.tex}, ainsi que les données contenues dans un fichier \texttt{.bib}.
L'application effectuerait ensuite selon les options différents traitement et afficherait le résultat sur la sortie standard ou un fichier.

\subsection{Structures de données utilisées}
\subsubsection{Collections}
Pour travailler, nous avons d'abord développé plusieurs structures de données génériques de type collection :
\begin{itemize}
 \item Les listes ordonnées 
 \item Les sets ordonnées (ainsi que son homologue non ordonné)
 \item Les maps sous la forme d'une liste ordonnée 
 \item Les tables de hachage
\end{itemize}

\subsubsection{Références}
Nous avons ensuite créer un type Références. Ces derniers représentent toutes les données nécessaires pour enregistrer une référence.
\end{document}